\documentclass[twocolumn]{article}

\usepackage{geometry}
\geometry{textwidth = 18cm,textheight = 24cm}

\usepackage{cite}
\usepackage{caption}
\usepackage{graphicx}
\usepackage{amsmath}
\usepackage{amssymb}
%\usepackage{braket}
\usepackage{textcomp}
%\usepackage{lmodern}
\usepackage{authblk}
\usepackage{datetime}
\usepackage{gensymb}
\usepackage{wrapfig}
%\usepackage[usenames,dvipsnames,svgnames,table]{xcolor}
\usepackage{booktabs}
%\usepackage{appendix}

%\usepackage[switch,columnwise]{lineno}
%\linenumbers

\newcommand{\onlinecite}[1]{\hspace{-1 ex} \nocite{#1}\citenum{#1}} 

\let\OLDthebibliography\thebibliography
\renewcommand\thebibliography[1]{
  \OLDthebibliography{#1}
  \setlength{\parskip}{0pt}
  \setlength{\itemsep}{0pt plus 0.3ex}
}
  
\title{Comparison of semiconducting and superconducting hardware for optoelectronic neuromorphic systems}
\author[1]{\Large{Bryce Primavera and Jeff Shainline}
\\
\textit{\large{National Institute of Standards and Technology}}
\\
\vspace{-0.2em}
\textit{\large{325 Broadway, Boulder, CO, USA, 80305}}
\\
%\vspace{-0.2em}
%\textit{\large{bryce.primavera@nist.gov, jeffrey.shainline@nist.gov}}
}
\date{\today}%\today

\begin{document}

\twocolumn[
  \begin{@twocolumnfalse}
    \maketitle
    \begin{abstract}

    \vspace{3em}
    \end{abstract}
  \end{@twocolumnfalse}
]

%\keywords{neural systems, neuromorphic computing, artificial intelligence, semiconductor devices, superconductor devices}

	
\setcounter{tocdepth}{3}
\setcounter{secnumdepth}{4}
\tableofcontents	
	
\section{\label{sec:introduction}Introduction}

\section{\label{sec:neural_device_requirements}Device requirements for neuromorphic systems}
	
\section{\label{sec:light_sources}Light Sources}
	
\section{\label{sec:detectors}Detectors}

\section{\label{sec:interconnection}Interconnection network and fan-out}

\section{\label{sec:synapses}Synaptic circuits and weighting}

\section{\label{sec:dendrites}Dendrites and fan-in}

\section{\label{sec:neurons}Neural integration and threshold}

\section{\label{sec:transmitters}Transmitter driver circuits}

\section{\label{sec:time_constants}Time constants and subthreshold oscillations}

\section{\label{sec:biasing}Biasing}

\section{\label{sec:systems}Power consumption, cooling, and system considerations}

%\begin{figure}[tb]
%    \centering{\includegraphics[width=8.6cm]{superconducting_gap.pdf}}
%	\captionof{figure}{\label{fig:superconducting_gap}Approximate normalized superconducting energy gap of niobium as a function of temperature.}
%\end{figure}

%\begin{equation}
%\label{eq:energy_gap}
%\frac{\Delta(T)}{\Delta(0)} \sim \bigg[1-\bigg(\frac{T}{T_c}\bigg)^{3.3}\bigg]^{1/2},
%\end{equation}


\section{Acknowledgements}

\vspace{1em}
\noindent This is a contribution of NIST, an agency of the US government, not subject to copyright.

%\clearpage
%\newpage
\appendix
%\appendixpage

\section{\label{apx:one}Appendix One}
Appendix One

\bibliographystyle{unsrt}	
\bibliography{ss_comparison}

\end{document}