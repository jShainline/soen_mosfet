\documentclass[conference]{IEEEtran}
\IEEEoverridecommandlockouts
% The preceding line is only needed to identify funding in the first footnote. If that is unneeded, please comment it out.
%\usepackage{cite}
\usepackage[noadjust]{cite}
\usepackage{amsmath,amssymb,amsfonts}
\usepackage{algorithmic}
\usepackage{graphicx}
\usepackage{textcomp}
\usepackage{xcolor}
\def\BibTeX{{\rm B\kern-.05em{\sc i\kern-.025em b}\kern-.08em
    T\kern-.1667em\lower.7ex\hbox{E}\kern-.125emX}}
\begin{document}


\title{Considerations for brain-scale artificial neural systems in semiconducting and superconducting optoelectronic hardware\\
\thanks{This work was performed under the following financial assistance award 70NANB18H006 from U.S. Department of Commerce, National Institute of Standards and Technology}
}


\author{\IEEEauthorblockN{Bryce Primavera}
\IEEEauthorblockA{\textit{Department of Physics} \\
\textit{University of Colorado Boulder}\\
\textit{National Institute of Standards and Technology}\\
Boulder, CO USA \\
bryce.primavera@nist.gov}
\and
\IEEEauthorblockN{Jeffrey Shainline}
\IEEEauthorblockA{\textit{Physical Measurement Laboratory} \\
\textit{National Institute of Standards and Technology}\\
Boulder, CO USA \\
jeffrey.shainline@nist.gov}
}


\maketitle

\begin{abstract}
The potential of optoelectronic hardware to achieve brain-scale neuromorphic computing is assessed. Both superconducting and semiconducting systems with direct optical synaptic connections are analyzed. The possibility of integrated memory, the potential for 3D integration, and the limits of optical communication for neural systems are highlighted.
\end{abstract}

\begin{IEEEkeywords}
neuromorphic, superconducting, 3D integration
\end{IEEEkeywords}

\section{Introduction}
The foundations of cognition remain a great frontier of science, with potentially enormous ramifications for technology and society. Leveraging certain principles of the information processing in the brain, chip manufacturers have begun development of low-power neuromorphic circuits for near-term deployment in edge AI applications \cite{merolla2014million, davies2018loihi}. Despite the exciting opportunities in CMOS electronic neural systems, physical limitations of electrical communication networks may not allow attainment of artificial neural systems of the same scale and complexity as the human brain, yet realization of such AI hardware may be necessary for testing theories of cognition, and might lead a platform that matches or even supersedes human intelligence.

With such lofty goals, it should not be surprising that the hardware for massive cognitive systems will deviate from current neuromorphic chips that target smaller scales. A significant challenge in scaling up neuromorphic architectures will be to construct systems that enable efficient communication with low latency amongst billions or trillions of neurons. Optical communication is well matched to the task, as the lack of resistive and capacitive parasitics makes optical links more amenable to the high fan-out required \cite{shainline2018largest}. In the present work, we consider aspects of optoelectronic hardware that must be considered if brain-scale systems are to be achieved. A variety of promising optoelectronic neuromorphic platforms have been proposed (\cite{shainline2019superconducting, nazirzadeh2018energy}), but here we limit the scope of the present study to systems that adhere to the following postulates:

\begin{enumerate}
    \item Direct, independent optical waveguide connections are made from neurons to synapses with no shared or multiplexed communication infrastructure.
    \item Optical communication from a neuron to each of its output synapses is binary and implemented with integrated light sources. 
    \item All synaptic, dendritic, and somatic computations are performed by electronic circuits.
\end{enumerate}

While direct connections sacrifice reconfigurability, the approach simplifies system overhead, eliminates traffic induced bottlenecks, and allows for separate, physical synaptic circuits to be implemented at every connection. In accordance with the final two principles, the synaptic circuits are tasked with converting digital optical spikes into electrical signals that can be weighted and summed by the somatic circuitry. It is our contention that performing these computations in the electrical domain will be preferable to the optical. Not only are nonlinearities much simpler to implement electrically, but by having dedicated receivers at every synapse, the parameters of each synapse (synaptic time-constant, plasticity hyperparameters, dynamics, etc.) can tuned individually. With these general principles, the neurons we consider consist of an optical transmitter whose signal is propagated to thousands of downstream connections via waveguides or fibers. Each synapse contains a photoreceiver and additional electronic circuitry for dynamics. 

Both superconducting and semiconducting platforms seem capable of supplying the necessary elements for the envisioned system. In the superconducting case, superconducting nanowire single photon detectors (SNSPDs) could function as extremely sensitive synaptic receivers. The semiconducting analogue would most likely be low-capacitance waveguide-integrated photodiodes, which have already been proposed and explored for intra-chip optical interconnectes. Either superconducting Josephson junctions (JJs) or MOSFETs are capable of implementing sophisticated synaptic and neuronal dynamics. We attempt to assess both the major points of contrast and shared challenges of the two possible routes.

\section{Optical Communication}
\subsection{Receivers}
Both superconducting and semiconducting systems provide mature technology for constructing optical receivers. SNSPDs, as the name implies, are capable of detecting light at the single photon level. For SNSPD-based synapses, the magnitude of an optical spiking event will be dictated by the shot-noise limit. In order to ensure 99\% probability of detection, synaptic pulses will require an average of 7 photons/spike to arrive at the detector. This corresponds to about 0.9 aJ at 1550 nm. However, superconducting electronics is always accompanied by a cooling overhead to keep the electronics below their critical temperature. At 4.2 K, the temperature of liquid helium, this adds a factor of $10^3$ to each synaptic pulse, meaning that no less than 1 fJ of energy will be needed per spiking event. Interestingly, this is quite close to the energy/bit required of ``receiverless" photodiodes proposed for short-distance optical communication in \cite{miller2017attojoule}. There it is argued that ultra-low capacitance photodiodes will enable communication at the scale of femtojoules per bit. It thus seems possible that the energy spent for communicating signals could be similar in both platforms.

\subsection{Transmitters}
While overall power consumption may be similar, the extreme sensitivity of SNSPDs removes significant burden from the optical transmitters located at each neuron. If the receiverless photodiode scheme is implemented, a superconducting link will require on the order of $10^3$ times less optical power than semiconducting optical link. This is particularly of consequence when neurons will be expected to fan-out to the levels observed in the human brain - roughly $10^3$-$10^4$ connections per neuron. A semiconductor platform would likely require a significant repeatering infrastructure or add gain stages at each synapse (which would drive up the overall energy cost per bit) in order to match the direct fan-out achievable with superconducting detectors. Additionally, operating at cryogenic temperatures allows for the possibility of silicon light sources, which would dramatically improve the feasibility of fabrication. Economical and scalable III/V integration would be a boon to either platform (especially for the semiconductor case), but has yet to be demonstrated despite decades of research. 

\section{Memory}
A local, nonvolatile memory technology will be key to the success of large-scale neuromorphic hardware - optoelectronic or not. There has been a huge thrust of research in this area, and several technologies have emerged as promising options for synaptic memory. RRAM, PCMs, ferroelectric transistors, spin-torque devices, and more could likely be integrated with CMOS neurons \cite{upadhyay2019emerging}. However, most of these devices are still in their infancy and suffer from significant questions about their variability, durability, and retention. Superconducting synaptic devices have also been proposed. Superconducting loop memory stores information as the magnitude of a persistent current circulating in a superconducting loop. Unlike many other technologies, programming pulses are identical to the spikes produced during normal network operation. The major drawback of superconducting memories is that they do not retain their state upon warming up the system to room temperature, making them at least semi-volatile. We do not attempt to discern the ideal memory for large-scale neuromorphic computing, but we do provide a list of benchmarks that a memory technology would ideally meet for use in brain-scale optoelectronic neuromorphic systems.

\begin{table}
  \begin{center}
    \label{tab:memory_metrics}
    \begin{tabular}{l|c|r} % <-- Alignments: 1st column left, 2nd middle and 3rd right, with vertical lines in between
      \textbf{Metric} & \textbf{Goal} \\
      \hline
      \textit{Endurance} & $>10^{11}$ updates \\
      \textit{Update Energy} & $<$ 30 pJ\\
      \textit{Update Speed} & $<100$ ns \\
      \textit{Weight Precision} & 4-10 bits
      
    \end{tabular}
    \caption{List of desired performance metrics for synaptic memory in a system with average fan-out of 1000, maximum spike rate of 10 MHz, average spike rate of 10 kHz, and spike energy of 10 fJ.}
  \end{center}
\end{table}

\section{3D Integration}
3D integration of photonic and electronic circuits has the potential to dramatically increase the number of neurons that can fit on a single wafer. Superconducting neural systems appear promising for 3D integration, given their low-temperature processing and recent demonstrations \cite{tolpygo2019planarized}. However, this field is yet to be extensively explored. 3D integration in CMOS platforms is challenged by the necessity of high-temperature processing steps, but bonding technology and new materials may one day solve this long elusive problem. Ultimately, any platform destined for extreme scaling will need to optimize performance while respecting some of these more subtle technological limitations.
\bibliographystyle{IEEEtran}
\bibliography{refs} % your .bib file


\section{Ease of Use}

\subsection{Maintaining the Integrity of the Specifications}

The IEEEtran class file is used to format your paper and style the text. All margins, 
column widths, line spaces, and text fonts are prescribed; please do not 
alter them. You may note peculiarities. For example, the head margin
measures proportionately more than is customary. This measurement 
and others are deliberate, using specifications that anticipate your paper 
as one part of the entire proceedings, and not as an independent document. 
Please do not revise any of the current designations.

\section{Prepare Your Paper Before Styling}
Before you begin to format your paper, first write and save the content as a 
separate text file. Complete all content and organizational editing before 
formatting. Please note sections \ref{AA}--\ref{SCM} below for more information on 
proofreading, spelling and grammar.

Keep your text and graphic files separate until after the text has been 
formatted and styled. Do not number text heads---{\LaTeX} will do that 
for you.

\subsection{Abbreviations and Acronyms}\label{AA}
Define abbreviations and acronyms the first time they are used in the text, 
even after they have been defined in the abstract. Abbreviations such as 
IEEE, SI, MKS, CGS, ac, dc, and rms do not have to be defined. Do not use 
abbreviations in the title or heads unless they are unavoidable.

\subsection{Units}
\begin{itemize}
\item Use either SI (MKS) or CGS as primary units. (SI units are encouraged.) English units may be used as secondary units (in parentheses). An exception would be the use of English units as identifiers in trade, such as ``3.5-inch disk drive''.
\item Avoid combining SI and CGS units, such as current in amperes and magnetic field in oersteds. This often leads to confusion because equations do not balance dimensionally. If you must use mixed units, clearly state the units for each quantity that you use in an equation.
\item Do not mix complete spellings and abbreviations of units: ``Wb/m\textsuperscript{2}'' or ``webers per square meter'', not ``webers/m\textsuperscript{2}''. Spell out units when they appear in text: ``. . . a few henries'', not ``. . . a few H''.
\item Use a zero before decimal points: ``0.25'', not ``.25''. Use ``cm\textsuperscript{3}'', not ``cc''.)
\end{itemize}

\subsection{Equations}
Number equations consecutively. To make your 
equations more compact, you may use the solidus (~/~), the exp function, or 
appropriate exponents. Italicize Roman symbols for quantities and variables, 
but not Greek symbols. Use a long dash rather than a hyphen for a minus 
sign. Punctuate equations with commas or periods when they are part of a 
sentence, as in:
\begin{equation}
a+b=\gamma\label{eq}
\end{equation}

Be sure that the 
symbols in your equation have been defined before or immediately following 
the equation. Use ``\eqref{eq}'', not ``Eq.~\eqref{eq}'' or ``equation \eqref{eq}'', except at 
the beginning of a sentence: ``Equation \eqref{eq} is . . .''

\subsection{\LaTeX-Specific Advice}

Please use ``soft'' (e.g., \verb|\eqref{Eq}|) cross references instead
of ``hard'' references (e.g., \verb|(1)|). That will make it possible
to combine sections, add equations, or change the order of figures or
citations without having to go through the file line by line.

Please don't use the \verb|{eqnarray}| equation environment. Use
\verb|{align}| or \verb|{IEEEeqnarray}| instead. The \verb|{eqnarray}|
environment leaves unsightly spaces around relation symbols.

Please note that the \verb|{subequations}| environment in {\LaTeX}
will increment the main equation counter even when there are no
equation numbers displayed. If you forget that, you might write an
article in which the equation numbers skip from (17) to (20), causing
the copy editors to wonder if you've discovered a new method of
counting.

{\BibTeX} does not work by magic. It doesn't get the bibliographic
data from thin air but from .bib files. If you use {\BibTeX} to produce a
bibliography you must send the .bib files. 

{\LaTeX} can't read your mind. If you assign the same label to a
subsubsection and a table, you might find that Table I has been cross
referenced as Table IV-B3. 

{\LaTeX} does not have precognitive abilities. If you put a
\verb|\label| command before the command that updates the counter it's
supposed to be using, the label will pick up the last counter to be
cross referenced instead. In particular, a \verb|\label| command
should not go before the caption of a figure or a table.

Do not use \verb|\nonumber| inside the \verb|{array}| environment. It
will not stop equation numbers inside \verb|{array}| (there won't be
any anyway) and it might stop a wanted equation number in the
surrounding equation.

\subsection{Some Common Mistakes}\label{SCM}
\begin{itemize}
\item The word ``data'' is plural, not singular.
\item The subscript for the permeability of vacuum $\mu_{0}$, and other common scientific constants, is zero with subscript formatting, not a lowercase letter ``o''.
\item In American English, commas, semicolons, periods, question and exclamation marks are located within quotation marks only when a complete thought or name is cited, such as a title or full quotation. When quotation marks are used, instead of a bold or italic typeface, to highlight a word or phrase, punctuation should appear outside of the quotation marks. A parenthetical phrase or statement at the end of a sentence is punctuated outside of the closing parenthesis (like this). (A parenthetical sentence is punctuated within the parentheses.)
\item A graph within a graph is an ``inset'', not an ``insert''. The word alternatively is preferred to the word ``alternately'' (unless you really mean something that alternates).
\item Do not use the word ``essentially'' to mean ``approximately'' or ``effectively''.
\item In your paper title, if the words ``that uses'' can accurately replace the word ``using'', capitalize the ``u''; if not, keep using lower-cased.
\item Be aware of the different meanings of the homophones ``affect'' and ``effect'', ``complement'' and ``compliment'', ``discreet'' and ``discrete'', ``principal'' and ``principle''.
\item Do not confuse ``imply'' and ``infer''.
\item The prefix ``non'' is not a word; it should be joined to the word it modifies, usually without a hyphen.
\item There is no period after the ``et'' in the Latin abbreviation ``et al.''.
\item The abbreviation ``i.e.'' means ``that is'', and the abbreviation ``e.g.'' means ``for example''.
\end{itemize}
An excellent style manual for science writers is \cite{b7}.

\subsection{Authors and Affiliations}
\textbf{The class file is designed for, but not limited to, six authors.} A 
minimum of one author is required for all conference articles. Author names 
should be listed starting from left to right and then moving down to the 
next line. This is the author sequence that will be used in future citations 
and by indexing services. Names should not be listed in columns nor group by 
affiliation. Please keep your affiliations as succinct as possible (for 
example, do not differentiate among departments of the same organization).

\subsection{Identify the Headings}
Headings, or heads, are organizational devices that guide the reader through 
your paper. There are two types: component heads and text heads.

Component heads identify the different components of your paper and are not 
topically subordinate to each other. Examples include Acknowledgments and 
References and, for these, the correct style to use is ``Heading 5''. Use 
``figure caption'' for your Figure captions, and ``table head'' for your 
table title. Run-in heads, such as ``Abstract'', will require you to apply a 
style (in this case, italic) in addition to the style provided by the drop 
down menu to differentiate the head from the text.

Text heads organize the topics on a relational, hierarchical basis. For 
example, the paper title is the primary text head because all subsequent 
material relates and elaborates on this one topic. If there are two or more 
sub-topics, the next level head (uppercase Roman numerals) should be used 
and, conversely, if there are not at least two sub-topics, then no subheads 
should be introduced.

\subsection{Figures and Tables}
\paragraph{Positioning Figures and Tables} Place figures and tables at the top and 
bottom of columns. Avoid placing them in the middle of columns. Large 
figures and tables may span across both columns. Figure captions should be 
below the figures; table heads should appear above the tables. Insert 
figures and tables after they are cited in the text. Use the abbreviation 
``Fig.~\ref{fig}'', even at the beginning of a sentence.

\begin{table}[htbp]
\caption{Table Type Styles}
\begin{center}
\begin{tabular}{|c|c|c|c|}
\hline
\textbf{Table}&\multicolumn{3}{|c|}{\textbf{Table Column Head}} \\
\cline{2-4} 
\textbf{Head} & \textbf{\textit{Table column subhead}}& \textbf{\textit{Subhead}}& \textbf{\textit{Subhead}} \\
\hline
copy& More table copy$^{\mathrm{a}}$& &  \\
\hline
\multicolumn{4}{l}{$^{\mathrm{a}}$Sample of a Table footnote.}
\end{tabular}
\label{tab1}
\end{center}
\end{table}

\begin{figure}[htbp]
\centerline{\includegraphics{fig1.png}}
\caption{Example of a figure caption.}
\label{fig}
\end{figure}

Figure Labels: Use 8 point Times New Roman for Figure labels. Use words 
rather than symbols or abbreviations when writing Figure axis labels to 
avoid confusing the reader. As an example, write the quantity 
``Magnetization'', or ``Magnetization, M'', not just ``M''. If including 
units in the label, present them within parentheses. Do not label axes only 
with units. In the example, write ``Magnetization (A/m)'' or ``Magnetization 
\{A[m(1)]\}'', not just ``A/m''. Do not label axes with a ratio of 
quantities and units. For example, write ``Temperature (K)'', not 
``Temperature/K''.

\section*{Acknowledgment}

The preferred spelling of the word ``acknowledgment'' in America is without 
an ``e'' after the ``g''. Avoid the stilted expression ``one of us (R. B. 
G.) thanks $\ldots$''. Instead, try ``R. B. G. thanks$\ldots$''. Put sponsor 
acknowledgments in the unnumbered footnote on the first page.

\section*{References}

Please number citations consecutively within brackets \cite{b1}. The 
sentence punctuation follows the bracket \cite{b2}. Refer simply to the reference 
number, as in \cite{b3}---do not use ``Ref. \cite{b3}'' or ``reference \cite{b3}'' except at 
the beginning of a sentence: ``Reference \cite{b3} was the first $\ldots$''

Number footnotes separately in superscripts. Place the actual footnote at 
the bottom of the column in which it was cited. Do not put footnotes in the 
abstract or reference list. Use letters for table footnotes.

Unless there are six authors or more give all authors' names; do not use 
``et al.''. Papers that have not been published, even if they have been 
submitted for publication, should be cited as ``unpublished'' \cite{b4}. Papers 
that have been accepted for publication should be cited as ``in press'' \cite{b5}. 
Capitalize only the first word in a paper title, except for proper nouns and 
element symbols.

For papers published in translation journals, please give the English 
citation first, followed by the original foreign-language citation \cite{b6}.

\begin{thebibliography}{00}
\bibitem{b1} G. Eason, B. Noble, and I. N. Sneddon, ``On certain integrals of Lipschitz-Hankel type involving products of Bessel functions,'' Phil. Trans. Roy. Soc. London, vol. A247, pp. 529--551, April 1955.
\bibitem{b2} J. Clerk Maxwell, A Treatise on Electricity and Magnetism, 3rd ed., vol. 2. Oxford: Clarendon, 1892, pp.68--73.
\bibitem{b3} I. S. Jacobs and C. P. Bean, ``Fine particles, thin films and exchange anisotropy,'' in Magnetism, vol. III, G. T. Rado and H. Suhl, Eds. New York: Academic, 1963, pp. 271--350.
\bibitem{b4} K. Elissa, ``Title of paper if known,'' unpublished.
\bibitem{b5} R. Nicole, ``Title of paper with only first word capitalized,'' J. Name Stand. Abbrev., in press.
\bibitem{b6} Y. Yorozu, M. Hirano, K. Oka, and Y. Tagawa, ``Electron spectroscopy studies on magneto-optical media and plastic substrate interface,'' IEEE Transl. J. Magn. Japan, vol. 2, pp. 740--741, August 1987 [Digests 9th Annual Conf. Magnetics Japan, p. 301, 1982].
\bibitem{b7} M. Young, The Technical Writer's Handbook. Mill Valley, CA: University Science, 1989.
\end{thebibliography}
\vspace{12pt}
\color{red}
IEEE conference templates contain guidance text for composing and formatting conference papers. Please ensure that all template text is removed from your conference paper prior to submission to the conference. Failure to remove the template text from your paper may result in your paper not being published.

\end{document}
